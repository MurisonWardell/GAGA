%\VignetteDepends{knitr}
%\VignetteIndexEntry{An introduction to GAGA}
%\VignetteCompiler{knitr}
%\VignetteEngine{knitr::knitr}

\documentclass{article}

% LaTeX packages 
% For hyperlinks
\usepackage{hyperref}

\usepackage{Sweave}
\begin{document}
\Sconcordance{concordance:GAGA_vignette.tex:GAGA_vignette.Rnw:%
1 54 1 50 0 6 1 5 0 53 1 18 0 39 1 5 0 17 1 11 0 7 1 6 0 7 1 6 0 %
28 1 7 0 37 1 13 0 9 1 6 0 15 1 6 0 42 1 9 0 59 1 5 0 55 1 26 0 7 %
1}



\title{An introduction to GAGA}
\author{Alex Murison and Christopher Wardell, alex.murison@icr.ac.uk}
\maketitle

% this is a comment.
This vignette serves as an introduction for the R package GAGA.  It covers the basic usage of the package and contains several worked examples.  If you use this package, please cite: (CITATION).

\paragraph{Installation:} The latest stable version can be installed from Bioconductor here...
 
The latest development version can be installed from our GitHub here: \href{https://github.com/MurisonWardell}{https://github.com/MurisonWardell}.

\begin{Schunk}
\begin{Sinput}
> ## Install
> source("http://bioconductor.org/biocLite.R")
> biocLite("GAGA")
> ## Load
> library(GAGA)
\end{Sinput}
\end{Schunk}

\pagebreak
\tableofcontents
\pagebreak

\section{Overview}
\subsection{Introduction}
Why we wrote gaga, the type of input data (SNVs - and reference our papers) and the basic outputs (phylogenies, proportions and heatmap)

\subsection{Genetic algorithms}
Overview of genetic algorithms and the string encoding each individual.
\subsection{GAGA input}
\subsection{GAGA output}

Discussion that you might get a different answer every time and that the number of clones is probably the most important variable.  The user MUST cycle through a number of clones and choose the lowest number of clones with the best answer.  INCLUDE SAMPLE CODE!


\section{Worked examples}
A number of sample data sets are distributed with the GAGA package and are discussed in order of increasing complexity.
\subsection{Example 1 - simple synthetic data}
A very small and simple synthetic data set is included.  To demonstrate that your GAGA installation is working, you can execute the following commands.  
\begin{Schunk}
\begin{Sinput}
> ## Load library
> library(GAGA)
> ## Load simple data set
> data("gaga_simple_data")
> ## There  are three time points (T0, T1 and T2)
> ## and four mutations (M1, M2, M3, M4)
> gaga_simple_data
\end{Sinput}
\begin{Soutput}
  names T0  T1  T2
1    M1  1 1.0 1.0
2    M2  0 0.5 1.0
3    M3  0 0.0 0.3
4    M4  0 0.0 0.5
\end{Soutput}
\end{Schunk}

\begin{Schunk}
\begin{Sinput}
> ## Execute gaga() function on the simple data set
> simpleDataSolution=gaga(gaga_simple_data, number_of_clones=4, nroot=1,iterations=3000)
\end{Sinput}
\end{Schunk}

The data represents 

\begin{Schunk}
\begin{Sinput}
> ## Execute gaga() function on the simple data set
> 
> #Top, zero-scoring solutions:
> #     x1 x2 x3 x4 x5 x6 x7 x8 x9 x10 x11 x12 x13 x14 x15 x16 x17 x18 x19 x20
> #[1,]  0  1  2  2 10  0  0  0  4   4   0   0   0   2   3   5   1   2   3   4
> #[2,]  0  1  2  2  8  0  0  0  4   4   0   0   0   2   3   5   1   2   3   4
> #[3,]  0  1  2  2 12  0  0  0  4   4   0   0   0   2   3   5   1   2   3   4
> #[4,]  0  1  2  2  9  0  0  0  4   4   0   0   0   2   3   5   1   2   3   4
> #[5,]  0  1  2  2 11  0  0  0  4   4   0   0   0   2   3   5   1   2   3   4
> #[6,]  0  1  2  2  9  0  0  0  3   3   0   0   0   2   3   5   1   2   3   4
\end{Sinput}
\end{Schunk}

\subsection{Example 2}
1.) Synthetic data as a proof of principle.  Note the inclusion of the jittered data (explain the jitter) and show that it works]
\subsection{Example 3}
2.) The yeast data, as it's REAL data and has a definite answer
\subsection{Example 4}
3.) Perhaps include the data from the recent LM paper?





\end{document}
